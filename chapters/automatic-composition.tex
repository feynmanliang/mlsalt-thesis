\documentclass[dissertation.tex]{subfiles}
\begin{document}

\chapter{Automatic composition}

Teacher forcing \cite{williams1989learning} preesents target outputs as
recurrent inputs rather than actual outputs since they are not yet correct
during training.

Differs from composition because sequential in nature: don't have the next note.

Can be described\cite{papadopoulos1999ai} as:

\begin{quote}
  a sequence (set) of rules (instructions, operations) for solving (accomplishing) a [particular] problem (task)
  [in a finite number of steps] of combining musical parts (things, elements) into a whole (composition)
\end{quote}

\cite{pearce2002motivations} emphasizes \emph{algorithmic composition} should ``be applied exclusively
to activities in which the motivations have this artistic character.''


\todo{Probabilistic piano roll: probability distributions over notes at each chord boundary over time}
\todo{Plot output unit activations over time, sum/average pool over all activations between chord boundary
delimiters: should yield a density over chords to be played for each time}
\todo{Plot hidden states over time, show that state trajectories suggest a use of history (how?)}

\section{Model description}

\subsubsection{Sequential processing and sampling}

We will train our RNNs to predict a distribution for the next character
$\x_{t+1}$ after the RNN has processed the sequence $\x_{1:t}$,
yielding a model which factorizes the sequence probability
\begin{equation}
    P(\x_{1:T}) = \prod_{t=1}^T P(\x_t | \x_{1:t-1} )
\end{equation}
Modeling this probability distribution over sequences is analogous to the
\textbf{language modeling} from speech recognition.

Note that the factorization of conditional distributions \emph{assumes a
sequential ordering in $t$}. This property is desirable as it enables sampling
from the model to generate new transcriptions by sampling from $P(\x_t |
\x_{1:t-1})$ at each timestep $t$ and using the sampled value as the next
input.

\todo{compare to bidirectional}.

\todo{for more details, see XYZ}.

\subsection{Parameter estimation of recursive neural networks}

\subsubsection{Modeling probabilities using the Boltzmann distribution and cross-entropy error criterion}

The parameters to the model are estimated in order to minimize the error
between the network outputs and provided labels. For language modeling, the
outputs $\y_t$ should parameterize a distribution for the next character
$P(\x_{t+1} | \x_{1:t})$. Suppose each input has finite support (i.e.
$\x_t \in [1,2,\cdots,K]$). If we choose the outputs to be $K$ dimensional
(i.e. $\y_t \in \RR^K$) then the \textbf{Boltzmann distribution}
(\autoref{eq:boltzmann-dist}) can be used:

\begin{equation}
    \label{eq:boltzmann-dist}
    P(\x_{t+1} = s | \x_{1:t})
    = \frac{\exp \left(-\y_{t,s}/T\right) }{ \sum_{k=1}^{K} \left(\exp -\y_{t,k}/T\right)}
\end{equation}

$T \in \RR^+$ is a \textbf{temperature} parameter \todo{relate to sampling}.

The labels provided are the actual next characters $\x_{t+1}$. Viewing
such labels as discrete probability distributions will all mass on a single atom,
one measure of difference between predictions and \todo{justify cross-entropy}

\section{Polyphonic modeling}

\section{Technical details}

\todo{We use teacher forcing \cite{williams1989learning}}

We construct multi-layer LSTM models with \texttt{num\_layers} number of
layers, each containing \texttt{rnn\_size} hidden units. The inputs $x_t$ are
one-hot-encoded before being passed through a \texttt{wordvec}-dimensional
vector-space embedding layer, which compresses the dimensionality down from
$|V| \approx 140$ to $\texttt{wordvec}$ dimensions. Dropout layers were added
between LSTM connections in both depth and time dimensions all with dropout
probability $\texttt{dropout} \in [0,1]$.

We build our models using the \texttt{torch7} framework and
an optimized implementation of LSTMs provided by \texttt{torch-rnn} \todo{cite}.

Models were trained using RMSProp \todo{cite} with batch normalization \todo{cite}
and an initial learning rate of $2 \times 10^{-3}$ decayed by $0.5$ every $5$
epochs. The back-propogation through time gradients were clipped
at $t$ \todo{cite Mikolov} and truncated after \texttt{seq\_length} time steps.
We use a mini-batch size of $50$.

\subsection{Multi-GPU implementation}

To accelerate model training, we parallelize models across multiple GPUs. This is possible
thanks to the summation operation in noisy gradient estimators:

\begin{equation}
  \frac{1}{N} \sum_{i=1}^N \nabla L_i(\theta) \approx \frac{1}{N} \sum_{i=1}^N \nabla L_i(\theta)
\end{equation}
\todo{Real citations on noisy gradient}

In particular, training RNNs with hidden state requires sequential traversal
of the dataset. Parallelizing sequential iteration is accomplished by first segmenting
into equal length segments and then initializing parallel iterators each
pointing at a different segment. Each iterator sequentially reads data
into GPU memory.

Model parameters are broadcast out to all GPUs on each \texttt{forward} pass
and gradients are accumulated during each \texttt{backward} pass.

Research in grid LSTMs suggests that we can go deeper by introducing
gates along the depth dimension to help permit information flow \todo{cite gird LSTMs}.

\printbibliography

\end{document}
