\chapter{Background}

% A more extensive coverage of what's required to understand your
% work. In general you should assume the reader has a good undergraduate
% degree in computer science, but is not necessarily an expert in
% the particular area you've been working on. Hence this chapter
% may need to summarize some ``text book'' material.

% This is not something you'd normally require in an academic paper,
% and it may not be appropriate for your particular circumstances.
% Indeed, in some cases it's possible to cover all of the ``background''
% material either in the introduction or at appropriate places in
% the rest of the dissertation.

Algorithmic composition is the application of a well-defined algorithmic
procedure to compose music.

An interesting question regarding creativity: if an algorithm faithfully reproduces
an artist's creative process, what is the difference between music produced by the artist
and music produced by the algorithm?

\section{Music theory}

A score of music is represented using a sequence of notes. Each note represents
a pitch (i.e.\ frequency) and duration (i.e.\ time interval).

When discussing pitches, it is common to refer to the difference between two
pitches as \textbf{pitch interval}s. One of the most fundamental pitch intervals is the
\textbf{octave} defined to be the interval between a frequency and its double.

While in theory an uncountable number of pitches are available, the tuning
system utilized by a piece of music oftentimes restricts the number of
available pitches. Western music commonly uses \textbf{twelve-tone equal
temperament} (12-TET) tuning, which divides an octave into 12 pitch classes
all equally spaced on a logarithmic scale. The interval between two adjacent
pitch classes (i.e. 1/12th an octave on log-scale) is called a \textbf{half step}
and two half steps are called a \textbf{whole step}

Tonal music is characterized by the prevalence of one pitch class (the
\textbf{tonic} around which the melody and harmony are built. A basic concept
within tonal music is the \textbf{scale} which defines a subset of pitch classes
that are ``in key'' with respect to the tonic. Two fundamental scales are the
major (with step pattern whole-whole-half-whole-whole-whole-half) and minor
scales (whole-half-whole-whole-half-whole-whole). The choice of tonic and scale
is referred to as the \textbf{key}and a change in key during a piece is called a
\textbf{modulation}Many musical phenomena such as stability, expectation, and
resolution can be attributed to tonal characteristics.

Note that while 12-TET restricts the possible intervals between notes, it does
not define an absolute reference pitch frequency. This degree of freedom gives
rise to transposition invariance: a score of music can be offset by a constant
pitch interval without affecting its tonal characteristics. For practical
performance purposes, the general tuning standard in modern times tunes the ``A''
note directly above ``middle C'' to 440 Hz (A440).

The \textbf{tempo} of a piece refers to its speed or pace and is measured by beats
per minute. In 4/4 time signature, a \textbf{quarter note} or \textbf{crotchet} denotes
the time interval between two beats. In addition to pitch quantization,
durations are also commonly quantized to subdivisions and multiples of a
crotchet.

\subsection{Notation}

We consider note duration, time, and velocity. We neglect changes in timing
(e.g. ritardandos), dynamics (e.g. crescendos), and stylistic notations (e.g.
accents, staccatos, legatos).

\emph{Piano roll} music transcriptions are quantized both in time ($t \in T$)
and note frequencies ($n \in N$). frequencies quantized to a piano roll.
\todo{Motivate quantization with Western music}.

We can represent a piano roll transcription as a high-dimensional vecctor
$X_{t,n} \in \RR^{|T| \times |N|}$ where $X_{t,n}$ denotes the note
velocities for note $n$ at time $t$.

